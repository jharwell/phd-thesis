%%%%%%%%%%%%%%%%%%%%%%%%%%%%%%%%%%%%%%%%%%%%%%%%%%%%%%%%%%%%%%%%%%%%%%%%%%%%%%%%
% app-exp-setup.tex: Experimental setup description.
% behavior
%%%%%%%%%%%%%%%%%%%%%%%%%%%%%%%%%%%%%%%%%%%%%%%%%%%%%%%%%%%%%%%%%%%%%%%%%%%%%%%%
\chapter{Experimental Setup}\label{app:exp-setup}

We use the ARGoS simulator~\cite{Pinciroli2012} with a dynamical physics model of
the marXbot (\cite{Dorigo2005c}) in a 3D space for maximum fidelity (robots are
still restricted to motion in the XY plane).

%%%%%%%%%%%%%%%%%%%%%%%%%%%%%%%%%%%%%%%%%%%%%%%%%%%%%%%%%%%%%%%%%%%%%%%%%%%%%%%%
\section{Foraging Task Decomposition Graph}\label{exp-setup:sec:foraging-tdgraph}
%%%%%%%%%%%%%%%%%%%%%%%%%%%%%%%%%%%%%%%%%%%%%%%%%%%%%%%%%%%%%%%%%%%%%%%%%%%%%%%%
%
\begin{figure}[!htbp]
  \centering
  \includegraphics[width=\textwidth]{\figroot{appendix}/tdgraph-foraging.png}
  \phdcaption{Task decomposition graph for the foraging task}{The set of red and
    blue tasks correspond to a scheme in which robots do the whole task or 1/2
    task (single decomposition option).  The set of red and blue and green tasks
    correspond to a scheme in which robots do the whole task, 1/2 task, or 1/4
    task (multi-decomposition option).  By decomposing the task into multiple
    parts, there are now task dependencies between the parts, which the swarm
    has to collectively learn.  Blue tasks ($\phi_i=1$) were defined
    in~\cite{Harwell2018}, green tasks ($\phi_i=2$) were
    defined~\cite{Harwell2019a}.\label{exp-setup:fig:tdgraph-foraging}}
\end{figure}
%
We instantiate the task decomposition graph shown
in~\cref{exp-setup:fig:tdgraph-foraging} for the experiments
in~\cref{chap:RQ1,chap:RQ2}, extending previous
work~\cite{Harwell2018,Pini2011b,Ferrante2015} which explicitly or implicitly
defined the \emph{Forager},\emph{Harvester}, and \emph{Collector} tasks
($\phi_0$,$\phi_1$) which together comprised $\TAGraphV{a}$ (red and green
vertices), described below.
%
\begin{itemize}
\item \emph{Forager}: Acquire a free block of highest utility and bring it to
  the nest. This is $\TAGraphRoot$, and one way of accomplishing the swarm
  objective $\SwarmObjective$ via $\phi_0$.

\item \emph{Harvester}: Acquire a free block of highest utility and bring it to
  the existing cache of highest utility, calculated as:
  %
  \begin{equation}\label{exp-setup:eqn:existing-cache-utility}
    \mu_{\hat{C_j}}(t) = \frac{e^{-\tau_{\hat{C_j}}(t)}{\lvert \hat{C_j} \rvert}}{\norm{\ArbRobot - \mathbf{C_j}}\norm{\mathbf{C_j} - \mathbf{nest}}}
  \end{equation}
  %
  where $\lvert \hat{C_j} \rvert$ is the estimated size of the $j$th cache known
  to $\ArbRobot$ (caches might not exist anymore, and therefore a robot only has
  estimates of their existence). This equation emphasizes selection of caches
  that are close to the position of $\ArbRobot$, but also that are closer to the
  nest, while accounting for the relevancy of a robot's information about
  $C_j$~\cite{Harwell2018}. This task accomplishes part of $\SwarmObjective$ via
  $\phi_1$.

\item \emph{Collector} Acquire a block from the existing cache of highest
  utility (Eqn.~\eqref{exp-setup:eqn:existing-cache-utility}) and bring it to
  the nest. This task accomplishes part of $\SwarmObjective$ via $\phi_1$.
\end{itemize}
%
We extend prior work with the following tasks, implemented as vertices
comprising the task sequence $\phi_2$ (green vertices
in~\cref{exp-setup:fig:tdgraph-foraging}):

\begin{itemize}
\item {\emph{Cache Starter}: Acquire a free block of highest utility and bring
    it partway back to the nest, and then drop it at a feasible site to start a
    \emph{De Novo} cache, according to~\cref{exp-setup:eqn:cache-site-select}. A
    feasible site is one at least a distance $\eta$ from all known caches
    $\hat{C_1}\ldots\hat{C_n}$.
    %
    The best cache site \textbf{x} for $\ArbRobot$ is computed as follows:
    %
    \arraycolsep=0.75pt
    \begin{equation}
      \label{exp-setup:eqn:cache-site-select}
      \begin{array}{lc}
        \underset{\mathbf{x}}{\max} & \mu_{\ArbRobot} \\
        \text{s.t.} & \norm{\mathbf{x} - \mathbf{\hat{C_{m}}}} \geq \eta, \; m = 1, \ldots, n. \\
      \end{array}
    \end{equation}
    where
    \begin{equation}
      \label{exp-setup:eqn:cache-site-utility}
      \mu_{\ArbRobot} = {\Big(\norm{\mathbf{x} -\mathbf{x}_{\ArbRobot}} \norm{\mathbf{x} - \frac{\mathbf{x}_{\ArbRobot} - \mathbf{nest}}{2}}\Big)}^{-1} \\
    \end{equation}

    % [JRH] see if I can find some natural analogues of this sort of utility--would
    % really strengthen my choice of utility function
    Intuitively, cache sites that are close to $\mathbf{x}_{\ArbRobot}$ are
    better (less work to get to, less chance of being found unsuitable upon
    arrival), as are sites that are close to the halfway point between the
    robot's current position and the nest (bisecting the space maximizes
    available areas for \emph{Collector} and \emph{Harvester} tasks, reducing
    interference). This task accomplishes part of $\SwarmObjective$ via
    $\phi_2$.}

\item {\emph{Cache Finisher}: Acquire a free block of highest utility and place
    it within a distance $\eta$ to a \emph{De Novo} cache (a single free block a
    distance $\eta$ from all other blocks) of maximum utility
    (\cref{exp-setup:eqn:existing-cache-utility} with
    $\lvert\hat{C_j}\rvert = 1$) in order to create a new cache. This task
    accomplishes part of $\SwarmObjective$ via $\phi_2$.}

\item {\emph{Cache Transferer}: Acquire a block from the existing cache of
    highest utility and transport it to the existing cache with the second
    highest utility. This accomplishes part of $\SwarmObjective$ via $\phi_2$.
  }
\item {\emph{Cache Collector}: Acquire a block from the existing cache of
    highest utility (\cref{exp-setup:eqn:existing-cache-utility}) and bring it
    to the nest. This task accomplishes part of $\SwarmObjective$ via $\phi_2$.}
\end{itemize}
%
By defining $\phi_2$, we have also implicitly defined $\beta_1,\beta_2$.

%%%%%%%%%%%%%%%%%%%%%%%%%%%%%%%%%%%%%%%%%%%%%%%%%%%%%%%%%%%%%%%%%%%%%%%%%%%%%%%%
\section{Foraging Scenarios}\label{exp-setup:sec:foraging-scenarios}
%%%%%%%%%%%%%%%%%%%%%%%%%%%%%%%%%%%%%%%%%%%%%%%%%%%%%%%%%%%%%%%%%%%%%%%%%%%%%%%%
%
We use the following experimental scenarios in the experiments
for~\gls{RQ1},~\gls{RQ2},~\gls{RQ3}.
%
\begin{figure}[!htbp]
  %
  \begin{minipage}{\textwidth}
    \includegraphics[width=\textwidth,height=0.22\textheight]{\figroot{appendix}/ss-foraging.png}
    \phdcaption{\emph{Single source} (SS) object distribution}{Objects on the
      right and the nest on the left\label{exp-setup:fig:fbd-ss}}
  \end{minipage}
  %
  \newline
  %
  \begin{minipage}{\textwidth}
    \includegraphics[width=\textwidth,height=0.22\textheight]{\figroot{appendix}/ds-foraging.png}
    \phdcaption{\emph{Dual source} (DS) object distribution}{Objects on the
      right and left and the nest in the center\label{exp-setup:fig:fbd-ds}}
  \end{minipage}
  %
  \newline
  %
  \begin{minipage}{0.49\textwidth}
    {\includegraphics[width=\textwidth,height=0.22\textheight]{\figroot{appendix}/pl-foraging.png}}
    \phdcaption{\emph{Power law } (PL) object distribution}{Nest in the
      center\label{exp-setup:fig:fbd-pl}}
  \end{minipage}
  %
  \begin{minipage}{0.49\textwidth}
    {\includegraphics[width=\textwidth,height=0.22\textheight]{\figroot{appendix}/rn-foraging.png}}
    \phdcaption{\emph{Random} (RN) object distribution}{Nest in the
      center\label{exp-setup:fig:fbd-rn}}
  \end{minipage}
  %
  \phdcaption{Example simulated foraging scenarios in ARGoS}{Multiple robots
    (blue blobs), objects to be collected (black squares), the nest in the
    center (gray), and the lights robots use for phototaxis and localization
    (yellow).\label{exp-setup:fig:fbd}}
\end{figure}

% [JRH] Show pictures of all the scenarios here, talk a little about each (e.g.,
% why PL is the most challenging scenario type).

\section{Source Code}\label{sec:exp-source-code}

Our open-source C++ code is available at the following locations:
%
\begin{itemize}
\item {\gls{RQ1},~\gls{RQ2},~\gls{RQ3}:
    https://github.com/swarm-robotics/fordyca }
\item {\gls{RQ4}: https://github.com/swarm-robotics/prism}
\end{itemize}
%
We also utilize SIERRA as an ARGOS driver framework to automatically generate
simulation inputs, run experiments, and generate the graphs seen in this thesis.
This is also open source, and available at:
%
\begin{itemize}
  \item {https://github.com/swarm-robotics/sierra}
  \item {https://github.com/swarm-robotics/titerra (SIERRA plugins)}
\end{itemize}

%%% Local Variables:
%%% mode: latex
%%% TeX-master: "../thesis"
%%% End:
