%%%%%%%%%%%%%%%%%%%%%%%%%%%%%%%%%%%%%%%%%%%%%%%%%%%%%%%%%%%%%%%%%%%%%%%%%%%%%%%
% intro.tex: Introduction to the thesis
%%%%%%%%%%%%%%%%%%%%%%%%%%%%%%%%%%%%%%%%%%%%%%%%%%%%%%%%%%%%%%%%%%%%%%%%%%%%%%%%
\chapter{Introduction}\label{chap:intro}
%%%%%%%%%%%%%%%%%%%%%%%%%%%%%%%%%%%%%%%%%%%%%%%%%%%%%%%%%%%%%%%%%%%%%%%%%%%%%%%%

% TRO
\gls{srs} is the study of the coordination of large numbers of simple
robots~\cite{Sahin2005}. \gls{srs} systems can be homogeneous (single robot type
and identical control software) or heterogeneous (multiple robot types and/or
control software)~\cite{Dorigo2013,Rizk2019,Ramachandran2020}. The main
differentiating factors between \gls{srs} systems and multi-agent robotics
systems stem from the mechanisms on which \gls{srs} systems are based.
Historically, these were principles of biological mimicry or problem solving
techniques inspired by natural systems of agents such as bees, ants, and
termites~\cite{Labella2006}. Moving beyond strictly biomimetic design, many
modern \gls{srs} systems have retained the desirable properties of natural
systems while employing elements of more conventional multi-robot system
design~\cite{Castello2016,Arvin2015,Steyven2018}.  First formally discussed
by~\cite{Winfield2005}, this discipline of \gls{swarm-engineering} seeks to
design systems which are based on rigorous mathematical methods but which also
possess the desirable system properties of emergent self-organization,
scalability, flexibility, and robustness, discussed
below~\cite{Brambilla2013a}.

\textbf{Emergent Self-Organization}. \emph{Self-organization} is the appearance
of structure within the swarm, which can be spatial, temporal, or functional. It
arises at the \emph{collective} level due to inter-robot interactions and robot
decisions at the \emph{individual} level~\cite{Winfield2005a,Galstyan2005}. Once
established, self-organization is generally resistant to external stimuli, and
this resistance is crucial to solving complex problems with only simple
agents~\cite{Hunt2020,DeWolf2005}.  Self-organization is related to, but
distinct, from the concept of \emph{emergence}, which is the set difference
between individual and collective swarm behavior, where the swarm behavior
arises both from inter-robot interactions and individual robot
decisions~\cite{Szabo2014,DeWolf2005}.  Emergent and self-organizing behaviors
are often both present in \gls{srs} systems as robots collectively find solutions
to a problem that they cannot solve alone~\cite{Cotsaftis2009,Hunt2020}. We term
this dichotomy \emph{emergent self-organization}, i.e, a two-way link between
collective and individual level behaviors.

\textbf{Scalability}. Scalable systems are able to maintain efficiency and
effectiveness as system size grows. \gls{srs} systems, like natural systems, can
achieve scalability to hundreds or thousands of agents through decentralized
control~\cite{Matthey2009}; other methods of maintaining effectiveness at larger
scales include local agent communication~\cite{Agassounon2001,Lerman2006}, and
heterogeneous robots or robot roles~\cite{Lu2020,Harwell2019a}.  Scalability
can arise directly from the design of the swarm control algorithm, but more
often as a cumulative result of emergent self-organizing behaviors, which
are themselves decentralized and local~\cite{DeWolf2005}.

\textbf{Flexibility}. Flexible systems are able to modify their collective
behavior in response to unknown stimuli in the external environment, e.g.,
changing weather conditions~\cite{Harwell2019a,Just2017,Hunt2020}. Individual
agents make decisions based on locally available information from neighbors and
their own limited sensor data. This produces a spatially distributed response
which arises from robot decisions as the swarm reacts and adapts to the
environment. \gls{srs} systems, like natural systems, can achieve flexibility in a
variety of ways: pheromone trails, site fidelity, localized communication, and
task allocation strategies~\cite{Just2017,Harwell2019a}. Some analytical methods,
such as task allocation strategies, explicitly encode mechanisms for the swarm
to collectively attempt to mitigate adversity and exploit beneficial changes in
dynamic environmental conditions~\cite{Just2017,Winfield2008}. Other methods,
such as pheromone trails, rely on the plasticity of self-organizing behaviors
and the development of emergent behaviors to achieve flexibility. The
flexibility of such methods can therefore be coupled to, but is still distinct
from, emergent and self-organizing behaviors~\cite{DeWolf2005}.

\textbf{Robustness}. Robust systems are able to modify their collective behavior
in response to internal, as opposed to environmental, stimuli.  Such stimuli can
include sensor and actuator noise, changes in system size due to the
introduction of new robots and robot failures.  Robustness is therefore an
emergent property of systems which demonstrate resilience to the effects of
internal stimuli on individual robots~\cite{DeWolf2005} (e.g., losing a single
robot or set of robots minimally perturbs the collective behavior of the swarm).
Robustness is crucial for crossing the simulation-reality
gap~\cite{Hecker2015,Francesca2014}.  Some \gls{srs} systems handle sensor and
actuator noise
analytically~\cite{Dallalibera2011,Claudi2014,Zong2006,Nurzaman2009,Turgut2008},
and can provide theoretical guarantees of robustness.  Other systems rely on
emergent behaviors~\cite{Harwell2020a}, and do not provide theoretical
guarantees; in such cases robustness is coupled to, but again distinct from,
emergent self-organization~\cite{Hunt2020,DeWolf2005}.

\section{Motivation and Investigative Contexts}

The duality between \gls{srs} and natural systems enables effective parallels to
be drawn with many naturally occurring problems, such as foraging, collective
transport of heavy objects, environmental monitoring, object sorting, hazardous
material cleanup, self-assembly, exploration, and collective decision
making~\cite{Hecker2015,Kumar2003,CarrilloZapata2020}.  As a result, \gls{srs}
systems are especially suited for complex tasks in dynamic environments where
robustness and flexibility are key to success, such as space, hazardous material
cleanup, fire and
rescue~\cite{Rouff2004,CarrilloZapata2020,Sahin2005,Flushing2014}, and
construction~\cite{Petersen2011}.

In this thesis, we contribute to the state of the art in~\gls{swarm-engineering}
by studying two important abstract scenarios which can be mapped to any of the
real-world use problems above: foraging and embodied graph manipulation.

\subsection{Foraging}
%
In a foraging task, robots gather objects (blocks) from across a finite
operating arena and bring them to a central location (often called the
\emph{nest}) for further processing (the \emph{central place foraging
  problem}). Foraging is one of the most studied applications of SR, due to its
straightforward mapping to real-world applications~\cite{Hecker2015}; for an
extensive discussion of the state of the art, see~\cite{Lu2020}.  The
complexity of the foraging task frequently gives rise to non-linear behavioral
dynamics due to inter-robot interactions, which in turn make designing~\gls{srs}
systems with provable bounds of behavior and practical utility difficult.

\subsection{Embodied Graph Manipulation}

\section{Summary of Contributions}
%
In the context of foraging and embodied graph manipulation, we study~\gls{srs}
systems from two perspectives. First, to understand from a mathematical point of
view how and why these properties arise in \gls{srs} systems. Second, to develop
theoretical tools to assist with designing \gls{srs} systems that provably
embody these properties.  In the remainder of this chapter, we outline our
contributions to each of these perspectives of \gls{srs} system research,
framing each contributions as answering an important open research question in
the field.

\subsection{Measurement of \glsfmtshort{srs} System Properties }
% 
When engineering practical~\gls{srs} systems, each of the following design
questions must be considered:
%
\begin{enumerate}
\item {Does the solution show emergent self-organization, indicating collective
    intelligence and its potential to be used in variants of the given problem?
  }
\item{Does the solution scale to the current and future needs of the modeled
    scenario?}
\item {Can the solution flexibly handle unknown environments or those with
    fluctuating conditions?}

\item {Is the solution robust to sensor/actuator noise and population size
    fluctuations?
  }
\end{enumerate}
%
In order to answer these questions we need a method for quantifying
each of the desirable system properties with numerical calculations; to the best
of our knowledge, such a methodology does not exist. This leads to the first
contribution of this thesis: the establishment of such a methodology by
answering the following fundamental research question:

\medskip\noindent
\gls{RQ1}: \glsdesc{RQ1}
\medskip

\noindent
In~\cref{chap:measurement} we present metrics for swarm emergent
self-organization, scalability, flexibility, and robustness (as analysis tools
to assist with answering our design questions. Our approach is domain agnostic,
and serves as a starting point to develop application-specific
variants. Furthermore, we establish the groundwork for more comprehensive
theories of swarm behavior, because measuring system properties precisely is a
necessary precursor to developing theories about what elements of a swarm
control algorithm give rise to observed behaviors. We apply our metrics to two
real-world problems: indoor warehouse object transport
(\cref{measurement:sec:sc1}) and outdoor search and rescue
(\cref{measurement:sec:sc2}).  Through application to these complex real-world
scenarios we expand the range of characteristics affecting swarm behavior which
can be meaningfully studied in simulation. For example, the ability to precisely
measure how different levels of sensor and actuator noise affect swarm behavior
allows us to incorporate noise-generating elements of real-world problems into
our scenario model and better study their effects. Without such methodology,
those effects can only be studied qualitatively, which is not as useful.

\subsection{Origin of Emergent Intelligence in \glsfmtshort{srs} Systems}
%
% [JRH] Need to check that the MRTA paper DOES use interdependent task sequences...
Of the four swarm properties described above, emergent self-organization is the
least understood and most difficult to quantify <REFS>. This lack of
understanding has hindered the development of~\gls{srs} systems because
designers are not able to guarantee that no negative behaviors (those that
oppose the system's goal) will emerge during system operation. Demystifying the
origin of emergent intelligence to help with the design of future~\gls{srs}
systems is therefore essential, and we consider the following important research
question:

\medskip\noindent
\gls{RQ2}: \glsdesc{RQ2}
\medskip

\noindent
Since answering this question in general for \gls{srs} systems is infeasible, if
not impossible, we consider it in the more limited, but still broadly applicable
context of~\gls{task-allocation}, by attempting to understand the origin of
emergent intelligence in robot swarms which perform task allocation. We use
definitions from the extended~\gls{mrta} taxonomy terminology proposed
by~\cite{Korsah2013}.  They separate the concept of an~\gls{atomic-task}, which
cannot be decomposed into smaller, simpler tasks, from that of
a~\gls{decomposable-task}, which can. They further define a~\gls{compound-task}
as a task which can be decomposed exactly one way, and a~\gls{complex-task},
which can be decomposed multiple ways.

Using these definitions, we can ask the following specific research questions to
answer~\gls{RQ2} in the context of task allocating swarms, the answers to which
form the second contribution of this thesis. First,

\medskip\noindent
\gls{RQ2.1}: \glsdesc{RQ2.1}
\medskip

\noindent
That is, do relational graphs with more vertices (tasks) and task dependencies
(edges) result in better collective task allocation decisions, and higher
measured levels of emergent intelligence?

To answer~\gls{RQ2.1}, we derive the algorithm~\gls{stochn1}, which uses the
neighborhood of a finished task within a task decomposition graph to
stochastically allocate a new task. We evaluate its emergent intelligence and
performance across \emph{compound} and \emph{complex} task decomposition graphs
for a foraging task, and show that swarm emergent intelligence is strongly
correlated with performance, and greater for complex than for compound task
decomposition graphs.

Second, 

\medskip\noindent
\gls{RQ2.2}: \glsdesc{RQ2.2}
\medskip

\noindent
That is, what is it that the swarm collectively learns from each relational
graph?  To answer~\gls{RQ2.2}, we derive~\gls{matopt}, a
matroid~\cite{Tutte1959,Whitney1935,Oxley2006} theoretic method in which we are
able to prove that if we disregard task dependencies from our task decomposition
graph, an extension of our task decomposition graph is optimally solvable with a
greedy algorithm for a single robot under some restrictions. It then follows
that an optimal allocation policy for the swarm is the disjoint union of
individual robot policies (intersection of matroids~\cite{Williams2017}). By
comparing the performance of~\gls{matopt} under constraints with that
of~\gls{stochn1}, we can determine whether emergent intelligence is more tied
to graph \emph{content} (tasks within the graph, treated as independent
by~\gls{matopt}), or to graph \emph{connectivity};~\gls{stochn1} should be the
highest performing allocation method if it is the former.

We compare the emergent intelligence and performance of~\gls{matopt}
against~\gls{stochn1} (which is specifically designed to enable collective
learning of graph connectivity, including task dependencies), and other state of
the art approaches at real-world scales (swarms of > 1,000 robots). We show that
swarm emergent intelligence is strongly tied to collective learning of graph
connectivity and structure (as opposed to graph content) by injecting accurate
knowledge about graph content (task costs), and comparing resulting
performance.~\gls{stochn1} is the most highly performing method in all tested
cases, providing strong quantitative evidence supporting the suitability of SR
systems for dangerous/unstable environments in which only partial or incomplete
information is available.~\gls{matopt} is shown to be suboptimal in many cases,
due to its disregard for graph structure and dependencies; however, results
suggest future synergies between theoretical methods leveraging emergent
intelligence is possible.

\subsection{Prediction and Control of \glsfmtshort{srs} Systems}

% [JRH] Talk about how putting bounds on swarm behavior is hard, but is
% necessary to do for practical systems. Existing work really on does this on
% trivial problems/single scenarios--can we do better?

It has been established that beyond a certain \emph{swarm density}
$\SwarmDensity$ (ratio of swarm size to arena
size)~\cite{Sugawara1997,Hamann2013}, interactions between components can
overtake outside interactions. The resulting non-linear behaviors which can
emerge are not generally predictable from component study; that is, directly
from the swarm control
algorithm~\cite{Cotsaftis2009,George2005,Hunt2020,DeWolf2005},
complicating~\gls{srs} system design. Hence, the collective performance of a
swarm $\TheSwarm$ of $\TheSwarmSize$ cooperating robots each running an
identical control algorithm $\kappa$ can be a non-linear function of the
behavior of a system of $\TheSwarmSize$ independent
robots~\cite{Harwell2020a}. Below this threshold, swarm behavior can be well
approximated using linear models. This leads to the third contribution of this
thesis: characterizing the \emph{practical} limits of linear modeling of the
non-linear behaviors of swarm, as a function of $\SwarmDensity$ by answering the
following research questions. First,

\medskip\noindent
\gls{RQ3.1}: \glsdesc{RQ3.1}
\medskip

\noindent
The exact value of $\SwarmDensity$ at which a given linear model of swarm
behavior breaks down is influenced by many factors, including the control
algorithm $\kappa$, the number of robots $\TheSwarmSize$, and characteristics of
the problem being solved, so in general it cannot be determined \emph{a priori}.

Second,

\medskip\noindent
\gls{RQ3.2}: \glsdesc{RQ3.2}
\medskip

\noindent

We seek answers to these questions for general foraging tasks, i.e., across
scenarios, scales, and swarm densities, and therefore require a model with wide
applicability to help answer them. Specifically, to answer~\gls{RQ3.1}, we must
evaluate the predictive power of a suitable model across scales and with
variable $\SwarmDensity$ in which we can reliably expect non-linear behaviors to
arise.

To the best of our knowledge a suitable model does not currently exist, and we
must develop one. The developed model improves on previous modeling
work~\cite{Lerman2002,Lerman2001,Lerman2003a} in two ways. First, while previous
work captured underlying swarm behaviors well, it relied on many free parameters
and extensive post-hoc model fitting to provide accurate predictions. We
eliminate most free parameters by deriving analytical expressions for them by
considering scenario characteristics, robotic control algorithm characteristics,
etc. The resulting model, while still requiring some minor post-hoc fitting, is
a step towards a true fundamental model of swarm behavior and an affirmative
answer to~\gls{RQ3.2}. Second, previous work only considered a single scenario
with small swarms; we consider multiple scenarios and both small and large
swarms. Our results in~\cref{ode-modeling:sec:results} show that our model
provides accurate predictions across block distributions, scales, and densities,
including some in which the swarm itself is exhibiting non-linear behaviors,
showing its broad utility in future SR system design.


%%% Local Variables:
%%% mode: latex
%%% TeX-master: "../thesis"
%%% End:
