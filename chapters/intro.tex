%%%%%%%%%%%%%%%%%%%%%%%%%%%%%%%%%%%%%%%%%%%%%%%%%%%%%%%%%%%%%%%%%%%%%%%%%%%%%%%
% intro.tex: Introduction to the thesis
%%%%%%%%%%%%%%%%%%%%%%%%%%%%%%%%%%%%%%%%%%%%%%%%%%%%%%%%%%%%%%%%%%%%%%%%%%%%%%%%
\chapter{Introduction}\label{chap:intro}
%%%%%%%%%%%%%%%%%%%%%%%%%%%%%%%%%%%%%%%%%%%%%%%%%%%%%%%%%%%%%%%%%%%%%%%%%%%%%%%%

% TRO
Swarm Robotics (SR) is the study of the coordination of large numbers of simple
robots~\cite{Sahin2005}. SR systems can be homogeneous (single robot type
and identical control software) or heterogeneous (multiple robot types and/or
control software)~\cite{Dorigo2013,Rizk2019,Ramachandran2020}. The main
differentiating factors between SR systems and multi-agent robotics
systems stem from the mechanisms on which SR systems are based.
Historically, these were principles of biological mimicry or problem solving
techniques inspired by natural systems of agents such as bees, ants, and
termites~\cite{Labella2006}. Moving beyond strictly biomimetic design, many
modern SR systems have retained the desirable properties of natural
systems while employing elements of more conventional multi-robot system
design~\cite{Castello2016,Arvin2015,Steyven2018}.  First formally discussed
by~\cite{Winfield2005}, this discipline of \gls{swarm-engineering} seeks to
design systems which are based on rigorous mathematical methods but which also
possess the desirable system properties of emergent self-organization,
scalability, flexibility, and robustness, discussed
below~\cite{Brambilla2013a}.

\textbf{Emergent Self-Organization}. \emph{Self-organization} is the appearance
of structure within the swarm, which can be spatial, temporal, or functional. It
arises at the \emph{collective} level due to inter-robot interactions and robot
decisions at the \emph{individual} level~\cite{Winfield2005a,Galstyan2005}. Once
established, self-organization is generally resistant to external stimuli, and
this resistance is crucial to solving complex problems with only simple
agents~\cite{Hunt2020,DeWolf2005}.  Self-organization is related to, but
distinct, from the concept of \emph{emergence}, which is the set difference
between individual and collective swarm behavior, where the swarm behavior
arises both from inter-robot interactions and individual robot
decisions~\cite{Szabo2014,DeWolf2005}.  Emergent and self-organizing behaviors
are often both present in SR systems as robots collectively find solutions
to a problem that they cannot solve alone~\cite{Cotsaftis2009,Hunt2020}. We term
this dichotomy \emph{emergent self-organization}, i.e, a two-way link between
collective and individual level behaviors.

\textbf{Scalability}. Scalable systems are able to maintain efficiency and
effectiveness as system size grows. SR systems, like natural systems, can
achieve scalability to hundreds or thousands of agents through decentralized
control~\cite{Matthey2009}; other methods of maintaining effectiveness at larger
scales include local agent communication~\cite{Agassounon2001,Lerman2006}, and
heterogeneous robots or robot roles~\cite{Lu2020,Harwell2019a}.  Scalability
can arise directly from the design of the swarm control algorithm, but more
often as a cumulative result of emergent self-organizing behaviors, which
are themselves decentralized and local~\cite{DeWolf2005}.

\textbf{Flexibility}. Flexible systems are able to modify their collective
behavior in response to unknown stimuli in the external environment, e.g.,
changing weather conditions~\cite{Harwell2019a,Just2017,Hunt2020}. Individual
agents make decisions based on locally available information from neighbors and
their own limited sensor data. This produces a spatially distributed response
which arises from robot decisions as the swarm reacts and adapts to the
environment. SR systems, like natural systems, can achieve flexibility in a
variety of ways: pheromone trails, site fidelity, localized communication, and
task allocation strategies~\cite{Just2017,Harwell2019a}. Some analytical methods,
such as task allocation strategies, explicitly encode mechanisms for the swarm
to collectively attempt to mitigate adversity and exploit beneficial changes in
dynamic environmental conditions~\cite{Just2017,Winfield2008}. Other methods,
such as pheromone trails, rely on the plasticity of self-organizing behaviors
and the development of emergent behaviors to achieve flexibility. The
flexibility of such methods can therefore be coupled to, but is still distinct
from, emergent and self-organizing behaviors~\cite{DeWolf2005}.

\textbf{Robustness}. Robust systems are able to modify their collective behavior
in response to internal, as opposed to environmental, stimuli.  Such stimuli can
include sensor and actuator noise, changes in system size due to the
introduction of new robots and robot failures.  Robustness is therefore an
emergent property of systems which demonstrate resilience to the effects of
internal stimuli on individual robots~\cite{DeWolf2005} (e.g., losing a single
robot or set of robots minimally perturbs the collective behavior of the swarm).
Robustness is crucial for crossing the simulation-reality
gap~\cite{Hecker2015,Francesca2014}.  Some SR systems handle sensor and
actuator noise
analytically~\cite{Dallalibera2011,Claudi2014,Zong2006,Nurzaman2009,Turgut2008},
and can provide theoretical guarantees of robustness.  Other systems rely on
emergent behaviors~\cite{Harwell2020a}, and do not provide theoretical
guarantees; in such cases robustness is coupled to, but again distinct from,
emergent self-organization~\cite{Hunt2020,DeWolf2005}.

\section{Motivation and Investigative Contexts}

The duality between SR and natural systems enables effective parallels to
be drawn with many naturally occurring problems, such as foraging, collective
transport of heavy objects, environmental monitoring, object sorting, hazardous
material cleanup, self-assembly, exploration, and collective decision
making~\cite{Hecker2015,Kumar2003,CarrilloZapata2020}.  As a result, SR
systems are especially suited for complex tasks in dynamic environments where
robustness and flexibility are key to success, such as space, hazardous material
cleanup, fire and
rescue~\cite{Rouff2004,CarrilloZapata2020,Sahin2005,Flushing2014}, and
construction~\cite{Petersen2011}.

In this thesis, we contribute to the state of the art in~\gls{swarm-engineering}
by studying two important abstract scenarios which can be mapped to any of the
real-world use problems above: foraging and~\gls{egm}.

In a foraging task, robots gather objects (blocks) from across a finite
operating arena and bring them to a central location (often called the
\emph{nest}) for further processing (the \emph{central place foraging
  problem}). Foraging is one of the most studied applications of SR, due to its
straightforward mapping to real-world applications~\cite{Hecker2015}; for an
extensive discussion of the state of the art, see~\cite{Lu2020}.  The complexity
of the foraging task frequently gives rise to non-linear behavioral dynamics due
to inter-robot interactions, which in turn make designing SR systems with
provable bounds of behavior and practical utility difficult.

In an~\gls{egm} task, robots interact with an environment which contains one or
more physically embodied structures, abstracted as heterogeneous grid graphs
(i.e., grid graphs in Euclidean space with potentially non-uniform distance
between nodes), and must manipulate the graph in some way to achieve a
collective goal. Each node in the graph is an object (block) which a robot can
carry, and each edge between two nodes $(u,v)$ corresponds to the extent of a
block anchored at $u$ along the X,Y, or Z axes to $v$.

\section{Summary of Contributions}
%
In the context of the foraging and~\gls{egm} problems, we study SR systems from
two perspectives. First, to understand from a mathematical point of view how and
why the properties of emergent self-organization, scalability, flexibility, and
robustness arise in SR systems, and how they can be leveraged to gain insight
into system behavior beyond raw performance observations. Second, to develop
theoretical tools to assist with designing SR systems that provably embody these
properties.  In the remainder of this chapter, we outline our contributions to
each of these perspectives of SR system research, framing each contribution as
contributing to an important open research question in the field.

\subsection{Measurement of SR System Properties }
%
When engineering practical SR systems, each of the following design
questions must be considered:
%
\begin{enumerate}
\item {Does the solution show emergent self-organization, indicating collective
    intelligence and its potential to be used in variants of the given problem?
  }
\item{Does the solution scale to the current and future needs of the modeled
    scenario?}
\item {Can the solution flexibly handle unknown environments or those with
    fluctuating conditions?}
\item {Is the solution robust to sensor/actuator noise and population size
    fluctuations?
  }
\end{enumerate}
%
In order to answer these questions we need a method for quantifying each of the
desirable system properties with numerical calculations; to the best of our
knowledge, such a methodology does not exist. This leads to the first
contribution of this thesis, published in~\cite{Harwell2021a}: the establishment
of such a methodology by answering the following fundamental research question:

\medskip\noindent
\gls{RQ1}: \textbf{\Glsdesc{RQ1}}
\medskip

\noindent
In~\cref{chap:RQ1} we present metrics for swarm emergent self-organization,
scalability, flexibility, and robustness (as analysis tools to assist with
answering our design questions. Our approach is domain agnostic, and serves as a
starting point to develop application-specific variants. Furthermore, we
establish the groundwork for more comprehensive theories of swarm behavior,
because measuring system properties precisely is a necessary precursor to
developing theories about what elements of a swarm control algorithm give rise
to observed behaviors. We apply our metrics to two real-world problems: indoor
warehouse object transport (\cref{RQ1:sec:sc1}) and outdoor search and rescue
(\cref{RQ1:sec:sc2}).  Through application to these complex real-world scenarios
we expand the range of characteristics affecting swarm behavior which can be
meaningfully studied in simulation. For example, the ability to precisely
measure how different levels of sensor and actuator noise affect swarm behavior
allows us to incorporate noise-generating elements of real-world problems into
our scenario model and better study their effects. Without such methodology,
those effects can only be studied qualitatively, which is not as useful.

\subsection{Origin of Emergent Intelligence in SR Systems}
%
% [JRH] Need to check that the MRTA paper DOES use interdependent task sequences...
Of the four swarm properties described above, emergent self-organization is the
least understood and most difficult to quantify <REFS>. This lack of
understanding has hindered the development of SR systems because
designers are not able to guarantee that no negative behaviors (those that
oppose the system's goal) will emerge during system operation. Demystifying the
origin of emergent intelligence to help with the design of future SR
systems is therefore essential, and we consider the following important research
question:

\medskip\noindent
\gls{RQ2}: \textbf{\Glsdesc{RQ2}}
\medskip

\noindent
Since answering this question in general for SR systems is infeasible, if
not impossible, we consider it in the more limited, but still broadly applicable
context of~\gls{task-allocation}, by attempting to understand the origin of
emergent intelligence in robot swarms which perform task allocation.

We compare the emergent intelligence and performance of~\gls{matopt}
against~\gls{stochn1} (which is specifically designed to enable collective
learning of graph connectivity, including task dependencies), and other state of
the art approaches at real-world scales (swarms of $>$ \unit{1,000} robots). We
show that swarm emergent intelligence is strongly tied to collective learning of
graph connectivity and structure (as opposed to graph content) by injecting
accurate knowledge about graph content (task costs), and comparing resulting
performance.~\gls{stochn1} is the most highly performing method in all tested
cases, providing strong quantitative evidence supporting the suitability of SR
systems for dangerous/unstable environments in which only partial or incomplete
information is available.~\gls{matopt} is shown to be suboptimal in many cases,
due to its disregard for graph structure and dependencies; however, results
suggest future synergies between theoretical methods leveraging emergent
intelligence is possible.

\subsection{Prediction and Control of the Average Behavior of SR Systems}
%
Many SR systems have been designed around imitating natural systems exactly, and
use heuristic decision making~\cite{Castello2016} rather than combining natural
principles with a strong mathematical
grounding~\cite{Talamali2020}. Nevertheless, heuristic approaches to swarm
control have been effective for robots that operate with incomplete information
and limited computing power. As a result, SR researchers average large numbers
of simulation runs to obtain empirical insights into real-world
problems~\cite{Harwell2019a} to develop accurate models of swarm behavior. The
emphasis on empirical rather than rigorous mathematical models has been the
chief impediment to a wider use of SR systems.  Systematically varying
individual agent parameters to study their effect on collective swarm behavior
is impractical, even in simulation. Mathematical characterization of collective
swarm behavior is more difficult, but provides the means to precisely predict it
\emph{a priori}---\emph{without} the need of repetitive experiments. Tools and
methods for such characterization are key to engineering better SR
systems, and motivates our investigation into the following research question,
and the third contribution of this thesis <Do I put ``under review'' here?>:

\medskip\noindent
\gls{RQ3}: \textbf{\Glsdesc{RQ3}}
\medskip

\noindent

Given the complexity of SR systems, and the frequently non-linear ways in
which behaviors can arise, it is difficult to obtain precise bounds on the
collective behavior of a swarm $\TheSwarm$ of $\TheSwarmSize$ robots each
running a control algorithm $\kappa$. Robots might need to respond to
environmental signals that arrive at unpredictable times; such systems are
well-modeled as asynchronous, and therefore difficult to predict precisely.
However, if we conceptualize $\TheSwarm$ as a differentiable, continuous
quantity, its dynamics can be modeled with~\glspl{ode} whose variables are the
population counts associated with different roles. We can apply a
macroscopic-continuous~\gls{ode} modeling approach for the \emph{average}
behavior of $\TheSwarm$ in the steady state~\cite{Berman2007}, with the caveat
that when using such a model to determine behavior of $\TheSwarm$, it is
possible that actual system behaviors are far from the
average~\cite{Lerman2004a}. Usually, the larger the system, the smaller the
fluctuations; in small systems the fluctuations resulting from \emph{finite size
  effects} can be of order $\TheSwarmSize$, resulting in models which are
accurate at asymptotically large scales but not small scales. The master
equation~\cite{VanKampen2007}, which is typically used to model expected average
behavior of systems, can be used to calculate the deviation from the average,
but such calculations are often intractable or algebraically difficult.

To answer~\gls{RQ3} for general foraging tasks, i.e., across scenarios, scales,
and swarm densities, we require a model with wide applicability, and must
evaluate the predictive power of a suitable model across scales and with
variable $\SwarmDensity$ in which we can reliably expect non-linear behaviors to
arise.  To the best of our knowledge a suitable model does not currently exist,
and we must develop one. The developed model improves on previous modeling
work~\cite{Lerman2002,Lerman2001,Lerman2003a} in two ways. First, while previous
work captured underlying swarm behaviors well, it relied on many free parameters
and extensive post-hoc model fitting to provide accurate predictions. We
eliminate most free parameters by deriving analytical expressions for them by
considering scenario characteristics, robotic control algorithm characteristics,
etc. The resulting model, while still requiring some minor post-hoc fitting, is
a step towards a true fundamental model of swarm behavior from first
principles. Second, previous work only considered a single scenario with small
swarms; we consider multiple scenarios and both small and large swarms. Our
results in~\cref{RQ3:sec:results} show that our model provides accurate
predictions across block distributions, scales, and densities, including some in
which the swarm itself is exhibiting non-linear behaviors, showing its broad
utility in future SR system design.

\subsection{Provably Correct Graph Manipulation Algorithms for SR Systems}

While the contributions from answering~\gls{RQ3.1} and~\gls{RQ3.2} advance the
state of the art in modeling of \emph{average} collective swarm behavior, no
guarantees are provided by the model that the behavior of individual robots will
not be far from the collective average at any given point. We begin to address
this gap with the fourth and final contribution of this thesis, by examining the
following research question:

\medskip\noindent
\gls{RQ4}: \textbf{\Glsdesc{RQ4}}
\medskip

\noindent
Similar to~\gls{RQ2}, answering~gls{RQ4} in general for SR systems is
infeasible, if not impossible, we again restrict our scope and consider it in
the more limited, but still broadly applicable context of the~\gls{egm}
problem. In the~\gls{egm} problem, agents must manipulate some aspect of their
environment, represented by a graph, by adding/removing nodes/edges over time in
order to achieve a goal. This abstraction maps naturally to two of the most
important real-world problems within which robotic automation has largely been
absent: construction and its counterpart, deconstruction~\cite{Werfel2006} (and
their associated variants: rubble clearing, pothole filling, etc.). Construction
and deconstruction tasks are inherently dangerous; e.g., constructing a home in
a stable environment such as a rural area has a fatal injury rate 3.5 times
higher than other occupations~\cite{Napp2012}. In more dangerous environments,
it is even higher, without any reduction in importance and/or urgency in
building important structures, including levee banks to prevent river flooding,
rail and power distribution networks (terrestrial or space), bases on the lunar
surface, homes in disaster areas~\cite{Zhang2011,Allwright2014,Grushin2006}, or
reinforcing infrastructure after natural disasters~\cite{Magnenat2012}.  With
the increased rise of extreme weather due to climate change, more and more
structures will need to be built, rebuilt, or have their rubble cleared away
quickly in potentially unstable or dangerous environments. Clearly, the
development of scalable autonomous~\gls{egm} systems is urgent, not only to
reduce risk to human life for construction workers, but also to more quickly
build simple structures to prevent future damage to infrastructure or loss of
life in affected areas.

Both construction and deconstruction (and~\gls{egm} tasks in general) are
challenging and complex problems, because successful systems must have the
ability to: (1) sense and manipulate assembly components; (2) interact with the
desired structure at all stages of the assembly/disassembly process; (3) satisfy
a variety of precedence constraints to ensure assembly/disassembly correctness;
and (4) ensure static stability and structural integrity throughout the
assembly/disassembly. The complexity of the~\gls{egm} problem, compounded with
robot perception, communication, and reliability issues in dangerous or unstable
environments means that conventional single-robot or multi-robot approaches
cannot easily be applied~\cite{Magnenat2012}. Nevertheless, some SR systems
system providing provable guarantees about robot behavior and convergence (e.g.,
finishing the assigned task) exist for the construction problem under strong
assumptions and restrictions~\cite{Petersen2011,Meng2008}; similarly for the
deconstruction problem~\cite{Petersen2011}.  Graph theoretic approaches to
the~\gls{egm} problem have also been successful, but have employed a centralized
controller~\cite{Worcester2011}; other works frequently use $\le 10$
robots~\cite{Worcester2011,Zhang2011,Allright2017,Zhang2008}.

As a result, while demonstrably successful approaches to small laboratory
variants of the~\gls{egm} problem, the wider applicability of SR systems to
real-world variants requiring hundreds or thousands of robots is unknown.
However, coordinating large numbers of embodied agents in such tasks at large
scales is clearly \emph{possible}: the success of natural systems in
collectively building structures of scope and size vastly greater than that of a
single worker is undeniable (e.g., ants, termites). Furthermore, the scope and
size of natural structures maps well to types of many artificial structures with
high utility that could be constructed (or deconstructed) by autonomous
multi-robot systems in real-world scenarios, providing strong natural
inspiration and motivation for this work.

We present a naturally inspired graph-theoretic model precisely defining a class
of graphs which can be provably manipulated to construct or deconstruct 3D
structures in different environments. Our graphical model has the following
properties:

\emph{Inherently parallel}. The addition or removal of nodes from graphs is
parallelized with $N/2$ simultaneous manipulations possible, where $N=min(X,Y)$
is the smallest 2D span of the graph. In previous works, the addition of nodes
is generally \emph{serial}; that is, regardless of the number of agents working
on a structure few, if any, simultaneous attachments or placements are
possible~\cite{Petersen2011,Werfel2011,Worcester2011}. Works allowing
simultaneous attachments and/or sub-structure assembly are usually simplified 2D
environments~\cite{Meng2008,Zhang2011,Matthey2009}, and the development and
evaluation of existing approaches under idealized conditions (typically a 2D or
3D discrete lattice with few constraints on motion, which is also discrete and
mostly random~\cite{Grushin2006,Theraulaz1995}) increases the simulation-reality
gap.

\emph{Embodiment Independence}. By representing the objects to be manipulated as
nodes within a graph, and obtaining specific properties of individual objects
from the graph structure, we obtain a general framework for reasoning about
discrete structures composed of (possibly) heterogeneous objects.

We also present a class of algorithms which guarantee collective completion of
the construction or deconstruction task as long as at least one robot remains
active, thereby decisively answering~\gls{RQ1} in the context of~\gls{egm}
problems. These algorithms do not require explicit inter-robot communication,
centralized control, or ``smart'' blocks~\cite{Werfel2006} to achieve
convergence, using joint environment manipulation to communicate and coordinate
and graphical invariants to guarantee task completion from any partially
constructed/deconstructed state.

This work is yet to be completed.


%%% Local Variables:
%%% mode: latex
%%% TeX-master: "../thesis"
%%% End:
