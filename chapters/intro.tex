%%%%%%%%%%%%%%%%%%%%%%%%%%%%%%%%%%%%%%%%%%%%%%%%%%%%%%%%%%%%%%%%%%%%%%%%%%%%%%%
% intro.tex: Introduction to the thesis
%%%%%%%%%%%%%%%%%%%%%%%%%%%%%%%%%%%%%%%%%%%%%%%%%%%%%%%%%%%%%%%%%%%%%%%%%%%%%%%%
\chapter{Introduction}\label{chap:intro}
%%%%%%%%%%%%%%%%%%%%%%%%%%%%%%%%%%%%%%%%%%%%%%%%%%%%%%%%%%%%%%%%%%%%%%%%%%%%%%%%

% TRO

\glspl{sr} is the study of the coordination of large numbers of simple
robots~\cite{Sahin2005}. \gls{sr} systems can be homogeneous (single robot type
and identical control software) or heterogeneous (multiple robot types and/or
control software)~\cite{Dorigo2013,Rizk2019,Ramachandran2020}. The main
differentiating factors between \gls{sr} systems and multi-agent robotics
systems stem from the mechanisms on which \gls{sr} systems are based.
Historically, these were principles of biological mimicry or problem solving
techniques inspired by natural systems of agents such as bees, ants, and
termites~\cite{Labella}.

Moving beyond strictly biomimetic design, many modern \gls{sr} systems have
retained the desirable properties of natural systems while employing elements of
more conventional multi-robot system
design~\cite{Castello2016,Arvin2015,Steyven2018}. Those properties are:

\textbf{Emergent Self-Organization}. \emph{Self-organization} is the appearance
of structure within the swarm, which can be spatial, temporal, or functional. It
arises at the \emph{collective} level due to inter-robot interactions and robot
decisions at the \emph{individual} level~\cite{Winfield2005a,Galstyan2005}. Once
established, self-organization is generally resistant to external stimuli, and
this resistance is crucial to solving complex problems with only simple
agents~\cite{Hunt2020,DeWolf2005}.  Self-organization is related to, but
distinct, from the concept of \emph{emergence}, which is the set difference
between individual and collective swarm behavior, which arises from inter-robot
interactions and individual robot decisions~\cite{Szabo2014,DeWolf2005}.
Emergent and self-organizing behaviors are often both present in \gls{sr} systems as
robots collectively find solutions to a problem that they cannot solve
alone~\cite{Cotsaftis2009,Hunt2020}. We term this dichotomy \emph{emergent
  self-organization}, i.e, a two-way link between collective and individual level
behaviors.

\textbf{Scalability}. Scalable systems are able to maintain efficiency and
effectiveness as system size grows. \gls{sr} systems, like natural systems, can
achieve scalability to hundreds or thousands of agents through decentralized
control~\cite{Matthey2009}; other methods of maintaining effectiveness at larger
scales include local agent communication~\cite{Agassounon2001,Lerman2006}, and
heterogeneous robots or robot roles~\cite{Lu2020,Harwell2019a}.  Scalability
can arise directly from the design of the swarm control algorithm, but more
often as a cumulative result of emergent self-organizing behaviors, which
are themselves decentralized and local~\cite{DeWolf2005}.

\textbf{Flexibility}. Flexible systems are able to modify their collective
behavior in response to unknown stimuli in the external environment, e.g.,
changing weather conditions~\cite{Harwell2019a,Just2017,Hunt2020}. Individual
agents make decisions based on locally available information from neighbors and
their own limited sensor data. This produces a spatially distributed response
which arises from robot decisions as the swarm reacts and adapts to the
environment. \gls{sr} systems, like natural systems, can achieve flexibility in a
variety of ways: pheromone trails, site fidelity, localized communication, and
task allocation strategies~\cite{Just2017,Harwell2019a}. Some analytical methods,
such as task allocation strategies, explicitly encode mechanisms for the swarm
to collectively attempt to mitigate adversity and exploit beneficial changes in
dynamic environmental conditions~\cite{Just2017,Winfield2008}. Other methods,
such as pheromone trails, rely on the plasticity of self-organizing behaviors
and the development of emergent behaviors to achieve flexibility. The
flexibility of such methods can therefore be coupled to, but is still distinct
from, emergent and self-organizing behaviors~\cite{DeWolf2005}.

\textbf{Robustness}. Robust systems are able to modify their collective behavior
in response to internal, as opposed to environmental, stimuli.  Such
stimuli can include sensor and actuator noise, changes in system size due to the
introduction of new robots and robot failures.
% As the swarm size fluctuates, robots may permanently or temporarily leave the
% swarm, as they experience hardware errors or task reallocations.
Robustness is therefore an emergent property of systems which demonstrate
resilience to the effects of internal stimuli on individual
robots~\cite{DeWolf2005} (e.g., losing a single robot or set of robots minimally
perturbs the collective behavior of the swarm).  Robustness is crucial for
crossing the simulation-reality gap~\cite{Hecker2015,Francesca2014}.  Some \gls{sr}
systems handle sensor and actuator noise
% [JRH] Thesis: ~\cite{Dallalibera2011,Claudi2014,Zong2006,Nurzaman2009,Turgut2008},
analytically~\cite{Dallalibera2011,Claudi2014,Turgut2008}, and can provide
theoretical guarantees of robustness.  Other systems rely on emergent
behaviors~\cite{Harwell2020a}, and do not provide theoretical guarantees; in
such cases robustness is coupled to, but again distinct from, emergent
self-organization. %~\cite{Hunt2020,DeWolf2005}.

The duality between designing swarm control algorithms directly to be scalable
and self-organizing, while incorporating emergent behaviors for flexibility and
robustness, can be leveraged in the design of practical \gls{sr}
systems.

% \emph{Swarm engineering} is a sub-field in swarm robotics for the systematic
% application of scientific and technical knowledge to model, design, validate,
% and operate a \gls{sr} system~\cite{Brambilla2013a}.
First formally discussed by~\cite{Winfield2005}, \emph{swarm engineering} seeks
to design systems which are based on rigorous mathematical methods but which
also possess the desirable system properties discussed
above~\cite{Brambilla2013a}.  Each of these desirable properties can be designed
for by answering the following design questions:
%
\begin{enumerate}
\item {Does the solution show emergent self-organization, indicating collective
    intelligence and its potential to be used in variants of the given problem?
  }
\item{Does the solution scale to the current and future needs of the modeled
    scenario?}
\item {Can the solution flexibly handle unknown environments or those with
    fluctuating conditions?}

\item {Is the solution robust to sensor/actuator noise and population size
    fluctuations? % as robots enter and leave the swarm over time?
  }
\end{enumerate}
%
In order to answer these questions we first need a method for quantifying each
of the desirable system properties with numerical calculations; such a
methodology does not exist, to the best of our knowledge. This leads to the
first contribution of this thesis: the establishment of such a methodology by
answering the following research question:

\noindent
\gls{RQ1}: \glsdesc{RQ1}

Of the for swarm properties described above, emergent self-organization is the
least understood <REFS>. In the second contribution of this thesis, we attempt
to understand its origin in the context of robot swarms which perform task
allocation. 
In this work, we seek to quantify
the relationship between swarm emergent intelligence and the richness of the
task decomposition graph used for allocation, and to understand the origin of
the emergent intelligence that arises from various task allocation
methods. Specifically, we seek
answers to the following questions:\\[-3ex]
%
\noindent
\gls{RQ2.1}: \glsdesc{RQ2.1}

\noindent
\gls{RQ2.2}: \glsdesc{RQ2.2}

% [JRH] Introduce task allocation perspective
We study emergent intelligence from a task allocation perspective...

% [JRH] Introduce foraging testbed

The duality between \gls{sr} and natural systems enables effective parallels to
be drawn with many naturally occurring problems, such as foraging, collective
transport of heavy objects, environmental monitoring, object sorting, hazardous
material cleanup, self-assembly, exploration, and collective decision
making~\cite{Hecker2015,Kumar2003,CarrilloZapata2020}.  As a result, \gls{sr}
systems are especially suited for complex tasks in dynamic environments where
robustness and flexibility are key to success, such as space, fire and
rescue~\cite{Rouff2004,CarrilloZapata2020}.

%
Our definitions of \emph{compound} and \emph{complex} task decomposition graphs
are drawn from the extended~\gls{mrta} taxonomy terminology proposed
by~\cite{Korsah2013}. They separate the concept of \emph{atomic} tasks, which
are not decomposable, from that of \emph{decomposable} tasks, which are. They
define \emph{compound} tasks as tasks having exactly one possible decomposition
(which is multi-agent allocatable), and \emph{complex} tasks, as tasks which
have multiple decompositions (each of which is multi-agent allocatable).


We study foraging from a task allocation perspective. In a foraging task, robots
gather objects from across a finite operating arena and bring them to a central
location (often called the \emph{nest}) for further processing. Foraging is one
of the most extensively studied applications of \gls{sr} due to its
straightforward mapping to important real-world applications~\cite{Hecker2015}
such as tracking lake health, clearing a corridor on a mining operation,
hazardous material cleanup, or search and
rescue~\cite{Sahin2005,Hecker2015,Labella}. 


% [JRH] Conclusion-y stuff at the end of intro

\begin{enumerate}
\item \textbf{Metrics for \gls{sr} systems properties}.  We present metrics for swarm
  emergent self-organization, scalability, flexibility, and robustness
  (\cref{sec:meas-emergence,sec:meas-scalability,sec:meas-flexibility,sec:meas-robustness})
  as analysis tools to assist with answering our design questions. Our approach
  is domain agnostic, and serves as a starting point to develop
  application-specific variants. Furthermore, we establish the groundwork for
  more comprehensive theories of swarm behavior, because precisely measuring
  system properties is a necessary precursor to developing theories about what
  elements of a swarm control algorithm give rise to observed behaviors.
\item \textbf{Realistic scenario modeling.}  We apply our metrics to two
  real-world problems: indoor warehouse object transport (\cref{sec:sc1}) and
  outdoor search and rescue (\cref{sec:sc2}).  Through application to these
  complex real-world scenarios we expand the range of characteristics affecting
  swarm behavior which can be meaningfully studied in simulation. For example,
  the ability to precisely measure how different levels of sensor and actuator
  noise affect swarm behavior allows us to incorporate noise-generating elements
  of real-world problems into our scenario model and better study their
  effects. Without such methodology, those effects can only be studied
  qualitatively, which is not as useful.
\end{enumerate}
%
% [JRH] Required by TRO: If a preliminary version of the paper has been (or is going
% to be) presented at conference(s), specify in a compact way which one(s); these
% conference papers should be explicitly cited in the present journal submission,
% with a clear statement describing the differences/changes/additions.  Papers
% failing to comply with this will be returned by the EiC.
%
A preliminary version of this paper was published in the IJCAI
conference~\cite{Harwell2019a}. It provided metrics for scabalility,
emergent self-organization, and flexibility, and evaluated them in an indoor
warehouse setting. This paper extends previous work in the following ways:
%
\begin{enumerate}
\item {Refined metrics for emergent self-organization, scalability,  and
    flexibility, and added metrics for robustness.}
\item {Applying our metrics to a search and rescue scenario, in addition to an
    indoor warehouse scenario.}
\item {Adding to the set of swarm control algorithms, which are used for
    comparison in each scenario, of two methods that use a different way of
    doing task allocation.}
\item {More thorough discussion of recent related work in swarm engineering and
    our metrics.  }
\end{enumerate}
%
The rest of the paper is organized as follows.~\cref{sec:related-work} discusses
related works in swarm engineering. We discuss swarm emergent self-organization,
scalability, flexibility, and robustness, and derive metrics for each
in~\cref{sec:meas-emergence,sec:meas-scalability,sec:meas-flexibility,sec:meas-robustness}.
\cref{sec:sc-overview} describes our two real-world problems along with design
constraints and candidate solutions. In \cref{sec:sc1,sec:sc2} we apply our
metrics to those two problems. We conclude with a discussion
in~\cref{sec:discussion} of common themes between the two scenarios.

% AAMAS
