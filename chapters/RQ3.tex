%%%%%%%%%%%%%%%%%%%%%%%%%%%%%%%%%%%%%%%%%%%%%%%%%%%%%%%%%%%%%%%%%%%%%%%%%%%%%%%%
% RQ3.tex: Chapter describing the ODE modeling approach to provable guarantees
% of swarm behavior.
%%%%%%%%%%%%%%%%%%%%%%%%%%%%%%%%%%%%%%%%%%%%%%%%%%%%%%%%%%%%%%%%%%%%%%%%%%%%%%%%
\chapter{RQ3: Prediction and Control of SR Systems}%
\label{chap:ode-modeling}
%%%%%%%%%%%%%%%%%%%%%%%%%%%%%%%%%%%%%%%%%%%%%%%%%%%%%%%%%%%%%%%%%%%%%%%%%%%%%%%%

% [JRH] This goes in the background here I think.

Many real problems across scales can be tackled using swarms in low density
environments (independent of swarm size) in which each robot is responsible for
a large area on the order of $100m^2$; these include indoor warehouses/object
transport ($\approx{64}m^2$), outdoor search and rescue, precision agriculture,
and field monitoring ($600,000m^2\approx{150}$ acres). While recent results
summarizing the challenges of moving swarms into the real world argue that
directing research towards low density swarm applications is
critical~\cite{Tarapore2020}, some researchers hold that swarms with low
$\SwarmDensity$ are not properly characterized as swarms, and are instead
systems of independent robots because they lack the high level of inter-agent
interaction which characterizes natural swarms. However, we argue that a high
level of inter-agent interaction (and therefore potentially emergent
self-organization) is only one of the defining properties of swarms. Low density
swarms exhibiting the properties of scalability, flexibility, and robustness can
properly still be considered swarms~\cite{Harwell2020a}.

%%% Local Variables:
%%% mode: latex
%%% TeX-master: "../thesis"
%%% End:
