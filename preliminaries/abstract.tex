%%%%%%%%%%%%%%%%%%%%%%%%%%%%%%%%%%%%%%%%%%%%%%%%%%%%%%%%%%%%%%%%%%%%%%%%%%%%%%%%
% abstract.tex: Abstract
%%%%%%%%%%%%%%%%%%%%%%%%%%%%%%%%%%%%%%%%%%%%%%%%%%%%%%%%%%%%%%%%%%%%%%%%%%%%%%%%

% Maria -- I think the abstract has to be just one page, but I am not sure
% TRO
Swarm Robotics is the study of the coordination of large numbers of simple
robots. In this thesis, we develop new mathematical tools and models to aid in
the design of swarm-robotic systems, and provide at least partial answers to
four important open research questions in the field.

First:~\glsdesc{RQ1} We present a set of metrics intended to
supplement designer intuitions when designing swarm-robotic systems, increase
accuracy in extrapolating swarm behavior from algorithmic descriptions and small
test experiments, and lead to faster and less costly design cycles. We build on
previous works studying self-organizing behaviors in autonomous systems to
derive a metric for swarm emergent self-organization.  We utilize techniques
from high performance computing, time series analysis, and queueing theory to
derive metrics for swarm scalability, flexibility to changing external
environments, and robustness to internal system stimuli such as sensor and
actuator noise and robot failures.  We demonstrate the utility of our metrics by
analyzing four different control algorithms in an indoor warehouse object
transport scenario with static objects and a spatially unconstrained outdoor
search and rescue scenario with moving objects. We show that our intuitions
about comparative algorithm performance are well supported by the quantitative
results obtained using our metrics, and that our metrics can be synergistically
used together to predict collective behaviors based on previous results in some
cases.

% In the spatially constrained warehouse scenario, efficient use of space
% is key to success so algorithms that use mechanisms for traffic regulation and
% congestion reduction are the most appropriate.  In the search and rescue
% scenario, the same will happen with algorithms which can cope well with object
% motion through dynamic task allocation and randomized search trajectories.

% AAMAS

Second:~\glsdesc{RQ2} We use task
allocation within the context of task decomposition graphs of different
cross-clique centralities, derive a self-organization task allocation algorithm
designed to be sensitive to clique centrality, and compare the results to other
state of the art algorithms. We derive a greedy algorithm that optimally solves
the swarm task allocation problem under some restrictive assumptions by
representing the swarm's task allocation space as a matroid. We compare the
greedy allocation method, which disregards task dependencies, with the
clique-sensitive method, which emphasizes collective learning of graph structure
(including dependencies). Results from an object gathering task show that swarm
emergent intelligence (1) is sensitive to cross-clique centrality of task
decomposition graphs (2) is positively correlated with performance, (3) arises
out of learning and exploitation of graph connectivity and structure, rather
than graph content.

% ODE/AR

Third:~\glsdesc{RQ3} We study the limits of linear modeling of swarm behavior by
characterizing the inflection point beyond which linear models of swarm
collective behavior break down. We design a linear model which strives to
capture the underlying dynamics of object gathering in robot swarms from first
principles, rather than extensively relying on post-hoc model fitting.  We
evaluate our model with swarms of up to 8,000 robots in simulation,
demonstrating that it accurately captures underlying swarm behavioral dynamics
when the swarm can be approximated using the mean-field model, and when it
cannot, and finite-size effects are present.  We further apply our model to
swarms exhibiting non-linear behaviors, and show that it still provides accurate
predictions in some scenarios, thereby establishing better practical limits on
linear modeling of swarm behaviors.

Fourth:~\glsdesc{RQ4} We model the interaction between a swarm and its
environment as a graph manipulation problem, and derive a provably correct
class of algorithm for incrementally constructing or deconstructing graphs which
certain properties.
