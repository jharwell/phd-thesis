%%%%%%%%%%%%%%%%%%%%%%%%%%%%%%%%%%%%%%%%%%%%%%%%%%%%%%%%%%%%%%%%%%%%%%%%%%%%%%%%
% abstract.tex: Abstract
%%%%%%%%%%%%%%%%%%%%%%%%%%%%%%%%%%%%%%%%%%%%%%%%%%%%%%%%%%%%%%%%%%%%%%%%%%%%%%%%

% AAMAS

We investigate the emergence of swarm intelligence using task allocation in
large robot swarms. First, we compare task decomposition graphs of different
levels of richness and measure the emergent intelligence arising from
self-organized task allocation by deriving \gls{stochn1}, a stochastic
allocation algorithm which contextualizes per-robot task allocation decisions
based on a previous task's neighborhood within the graph. The results are
compared to other state of the art algorithms. Second, we derive \gls{matopt}: a
greedy algorithm that optimally solves the swarm task allocation problem by
representing the swarm's task allocation space as a matroid under some
restrictive assumptions. We compare the \gls{matopt} allocation method, which
disregards task dependencies, with \gls{stochn1}, which emphasizes collective
learning of graph structure (including dependencies). Results from an object
gathering task show that swarm emergent intelligence (1) is sensitive to the
richness of task decomposition graphs (2) is positively correlated with
performance, (3) arises out of learning and exploitation of graph connectivity
and structure, rather than graph content.


% TRO

We present a set of metrics intended to supplement designer intuitions when
designing swarm-robotic systems, increase accuracy in extrapolating swarm
behavior from algorithmic descriptions and small test experiments, and lead to
faster and less costly design cycles. We build on previous works studying
self-organizing behaviors in autonomous systems to derive a metric for swarm
emergent self-organization.  We utilize techniques from high performance
computing, time series analysis, and queueing theory to derive metrics for swarm
scalability, flexibility to changing external environments, and robustness to
internal system stimuli such as sensor and actuator noise and robot failures.

We demonstrate the utility of our metrics by analyzing four different control
algorithms in an indoor warehouse object transport scenario with static objects
and a spatially unconstrained outdoor search and rescue scenario with moving
objects. In the spatially constrained warehouse scenario, efficient use of space
is key to success so algorithms that use mechanisms for traffic regulation and
congestion reduction are the most appropriate.  In the search and rescue
scenario, the same will happen with algorithms which can cope well with object
motion through dynamic task allocation and randomized search trajectories. We
show that our intuitions about comparative algorithm performance are well
supported by the quantitative results obtained using our metrics, and that our
metrics can be synergistically used together to predict collective behaviors
based on previous results in some cases.
