%%%%%%%%%%%%%%%%%%%%%%%%%%%%%%%%%%%%%%%%%%%%%%%%%%%%%%%%%%%%%%%%%%%%%%%%%%%%%%%%
% Term Definitions
%%%%%%%%%%%%%%%%%%%%%%%%%%%%%%%%%%%%%%%%%%%%%%%%%%%%%%%%%%%%%%%%%%%%%%%%%%%%%%%%
\newglossaryentry{atomic-task} {
  %
  name={atomic task},
  %
  description={a task which cannot be decomposed into smaller, simpler tasks}
}

\newglossaryentry{task-partitioning} {
  %
  name={task partitioning},
  %
  description={the process of dividing a~\gls{decomposable-task} into simple
    subtasks which are multi-agent allocatable\cite{Ratnieks1999, Korsah2013}}
}

\newglossaryentry{decomposable-task} {
  %
  name={decomposable task},
  %
  description={a task which can be decomposed into smaller, simpler tasks}
}

\newglossaryentry{compound-task}
{
  %
  name={compound task},
  %
  description={a task which has exactly one possible decomposition into
    a sequence of interdependent subtasks}
}

\newglossaryentry{complex-task}
{
  %
  name={complex task},
  %
  description={a task which can be decomposed in multiple ways into multiple
    sequences of interdependent subtasks
  }
}

\newglossaryentry{compound-tdgraph}
{
  %
  name={compound task decomposition graph},
  %
  description={the graph }
}

\newglossaryentry{swarm-engineering}
{
  %
  name={swarm engineering},
  %
  description={a sub-field of SR system design which seeks to
    design systems which are based on rigorous mathematical methods but which
    also possess the desirable system properties of emergent self-organization,
    scalability, flexibility, and robustness } }

\newglossaryentry{task-allocation}
{
  %
  name={task allocation},
  %
  description={the process of choosing which task to do next after a given task
    has finished. The entity allocates the task does not have to be the same as
    the entity which executes it (e.g., a scheme in which a centralized
    controller performs all task allocations and then assigns robots to tasks)
  }
}

\newglossaryentry{egm}
{
  %
  name={embodied graph manipulation},
  %
  description={the process of manipulating some aspect of the physical
    environment, represented as a graph, by adding or removing nodes or edges in
    the graph.}
}
